\documentclass[11pt,wide]{mwart}

\usepackage[T1]{fontenc}
% Można też użyć UTF-8
\usepackage[utf8]{inputenc}

% Język
\usepackage[polish]{babel}
% \usepackage[english]{babel}

% Rózne przydatne paczki:
% - znaczki matematyczne
\usepackage{amsmath, amsfonts}
% - wcięcie na początku pierwszego akapitu
\usepackage{indentfirst}
% - komenda \url
\usepackage{hyperref}
% - dołączanie obrazków
\usepackage{graphics}
% - szersza strona
\usepackage[nofoot,hdivide={2cm,*,2cm},vdivide={2cm,*,2cm}]{geometry}
\frenchspacing


\usepackage{hyperref}
\usepackage{url}

\newtheorem{tw}{Twierdzenie}
\newtheorem{alg}{Algorytm}

\date{Wrocław, \today}

\title{\Large\textbf{Podstawowe umiejętności w \LaTeX} }
\author{Franciszek Zdobylak \thanks{\textit{E-mail}: \texttt{310313@uwr.edu.pl}}}

\begin{document}

\maketitle
\tableofcontents

\section{Umiejętność tworzenia działów}
Umiejętność tworzenia działów jest jedną z podstawowych umiejętnośći potrzebnych do pisania
sensownych prac. W ten sposób porządkujemy to co chcemy przekazać czytelnikowi. 
W tym rozdziale przedstawię parę sposobów na tworzenie podrozdziałów.

\subsection{Rozdział z numerem}
Rozdziały z mumerem pomagają w odnalezieniu się podczas czytania.
\subsection*{Rozdział bez numeru}
Rozdziały bez numerów pomagają nam zwrócić uwagę na ważne zagadnienia związane z omawianym tematem.
\subsubsection{Podpodrozdział}
Takie zagnieżdżone rozdziały przydają się gdy chcemy bardziej zgłębić dany temat.

\section{Notacja matematyczna}

Notacja matematyczna bardzo przydaje się podczas naszych studiów. Na przykład do zapisania
twierdzenia Taylora.

\begin{tw}["Twierdzenie Taylora"]
  Niech $Y$ będzie przestrzenią unormowaną oraz $f: [a,b] \rightarrow Y$ będzie funkcją
  $(n+1)$ razy różniczkowalną na przedziale $[a,b]$ w sposób ciągły. 
  Wówczas dla każdego punktu $x$ z przedziału $(a,b)$ spełniony jest wzór
  $$ f(x) = f(a) + \frac{x - a}{1!} f^{(1)}(a) + \frac{(x-a)^2}{2!} f^{(2)}(a) + \dots
    + \frac{(x-a)^n}{n!} f^{(n)}(a) + R_n(x, a)$$
  gdzie $f^{(k)}$ jest pochodną k-tego stopnia funkcji $f$, a $R_n(x,a)$ spełnia warunek
  $$ \lim_{x\rightarrow a}{\frac{R_n(x,a)}{(x - a)^n}} = 0$$
\end{tw}

\section{Tabele}
Kolejnym ważnym narzędziem przydatnym przy pisaniu naukowych artykułów są tabele.
\begin{center}
\begin{tabular}{|c|r|}
  \hline
  \textbf{Nazwa} & \textbf{Symbol}\\
  \hline
  Jeden & $1$\\
  \hline
  Dwa & $2$\\
  \hline
  Trzy & $3$\\
  \hline
\end{tabular}
\end{center}
\end{document}
